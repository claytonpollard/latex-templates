\documentclass[aspectratio=169,10pt]{beamer}

% Mathematics
\usepackage{amsmath, amsfonts, mathtools, amsthm, amssymb}
\usepackage{mathrsfs}
\usepackage{cancel}

% Colours
\usepackage[usenames,dvipsnames]{xcolor}
  \definecolor{col1}{HTML}{3A86FF}
  \definecolor{col2}{HTML}{0BBF7D}
  \definecolor{col3}{HTML}{FFBE0B}
  \definecolor{col4}{HTML}{FF006E}
  \definecolor{col5}{HTML}{733907}
  \definecolor{col6}{HTML}{B340D7}

% Fonts
\usepackage{libertine}
\usepackage{newtxmath}
%\usepackage{fontspec}
%\setmonofont[Scale=MatchLowercase]{Courier New Bold}

% Formatting
\usepackage{paralist}
\usepackage{enumitem}
\setlist[itemize]{itemsep=2pt, topsep=2pt}
\usepackage{marginnote}
\renewcommand*{\marginfont}{\footnotesize}
\usepackage[top=1.5cm,bottom=1.5cm,inner=2.5cm,outer=4cm,marginparwidth=2.5cm]{geometry}
\usepackage{hyperref}
\usepackage{titlesec}
\setlength\parindent{0pt}
\pagenumbering{gobble}
\usepackage{multicol}
\tolerance=1
\emergencystretch=\maxdimen
\hyphenpenalty=10000
\hbadness=10000

% Graphics
\usepackage{graphicx}
\graphicspath{{graphics/}}

% Honestly don't remember
\usepackage[normalem]{ulem}
\usepackage[misc]{ifsym}

% Chapter and section formatting
\titleformat{\section}[display]{\Large\sffamily\bfseries}{}{.5ex}{\vspace{-1em}}[\vspace{-0.5ex}\rule{\textwidth}{0.3pt}]
\titleformat{\subsection}[display]{\Large\sffamily\bfseries}{}{.5ex}{\vspace{-1em}}
\titleformat{\subsubsection}[display]{\sffamily\bfseries}{}{.5ex}{\vspace{-1em}}

% Title
\usepackage{titling}
\setlength{\droptitle}{-2em}
\pretitle{\begin{Huge}\bfseries\sffamily}
\posttitle{\end{Huge}\vspace{-4em}
}
% Grid Background
%%%`\GridOn' to enable on page.

\usepackage[pages=some]{background}
\newcommand{\GridOn}{\BgThispage}
\def\Step{5mm}
\backgroundsetup{
scale=1,
angle=0,
contents={
\begin{tikzpicture}[remember picture,overlay]
\draw[step=\Step,help lines,dashed]
  (current page.north west) grid  (current page.south east);
\draw[step=2*\Step,help lines]
  (current page.north west) grid  (current page.south east);
  \end{tikzpicture}
  }
}

% Cloze
%%% `\ShowClozetrue' to enable, `\ShowClozefalse' to disable
\newif\ifShowCloze
\ShowClozefalse
\newcommand{\Cloze}[1]{
    \ifShowCloze
        \dotuline{\parbox[b]{\widthof{\LARGE\textbf{#1}}}{ \vphantom{\vspace{1ex}} \centering %
        {\color{OrangeRed}  {\protect #1}} %
        }}
    \else
        \dotuline{\parbox[b]{\widthof{\LARGE\textbf{#1}}}{ \vphantom{\vspace{1ex}}\centering %
            \hfill
            }
        }
    \fi
}

% Boxes
\usepackage[most]{tcolorbox} % Required for boxes
	\tcbuselibrary{skins,breakable,xparse}
\usepackage{varwidth}

\newtcolorbox[auto counter]{definition}[1][]{standard jigsaw,enhanced,sharp corners,frame hidden,boxrule=0pt,breakable,colback=col1!20!white,fonttitle=\bfseries,coltitle=col1!50!black,colframe=col1!50!white,title=Definition~\thetcbcounter\quad#1\newline,attach title to upper,borderline west={2pt}{0pt}{col1!80!black},left=3mm}

\newtcolorbox[auto counter]{examplebox}[1][]{standard jigsaw,enhanced,sharp corners,frame hidden,boxrule=0pt,breakable,colback=col2!20!white,fonttitle=\bfseries,coltitle=col2!50!black,colframe=col2!50!white,title=Example~\thetcbcounter\quad#1\newline,attach title to upper,borderline west={2pt}{0pt}{col2!80!black},left=3mm,bottom=3mm,borderline south={1pt}{0pt}{col2!80!black}}

\newtcolorbox{solutionbox}[1][height=4cm]{standard jigsaw,enhanced,sharp corners,frame hidden,boxrule=0pt,breakable,#1,colback=col2!50!white,fonttitle=\bfseries,coltitle=col2!50!black,colframe=col2!50!white,opacityback=.1,borderline west={2pt}{0pt}{col2!80!black},left=3mm,top=3mm,before={\vspace{-5pt}}}

\newtcolorbox{note}{standard jigsaw,enhanced,sharp corners,frame hidden,boxrule=0pt,breakable,colback=col3!20!white,fonttitle=\normalfont\bfseries,coltitle=col3!50!black,colframe=col3!50!white,title=Note~,attach title to upper,borderline west={2pt}{0pt}{col3!80!black},left=3mm,fontupper=\itshape}

\newtcolorbox{important}{standard jigsaw,enhanced,sharp corners,frame hidden,boxrule=0pt,breakable,colback=col4!20!white,fonttitle=\bfseries,coltitle=col4!50!black,colframe=col4!50!white,title=Important note\newline,attach title to upper,borderline west={2pt}{0pt}{col4!80!black},left=3mm}

\newtcolorbox{further}{standard jigsaw,enhanced,sharp corners,frame hidden,boxrule=0pt,breakable,colback=col5!20!white,fonttitle=\bfseries,coltitle=col5!50!black,colframe=col5!50!white,title=Further exercises\newline,attach title to upper,borderline west={2pt}{0pt}{col5!80!black},left=3mm}

\newtcolorbox{outcome}{standard jigsaw,enhanced,sharp corners,frame hidden,boxrule=0pt,breakable,colback=col6!20!white,fonttitle=\bfseries,coltitle=col6!50!black,colframe=col6!50!white,title=Learning Outcome\newline,attach title to upper,borderline west={2pt}{0pt}{col6!80!black},left=3mm}

\usepackage{environ}

\NewEnviron{example}[2][height=4cm]{
  \begin{examplebox}[#2]
  \BODY
  \end{examplebox}
  \begin{solutionbox}[#1]
  \end{solutionbox}
}

\newenvironment{answers}{
\subsubsection{Answers}
\scriptsize
}
{ \normalsize
}

\title{Earning income: Salary}
\subtitle{Financial Mathematics}
%\author{}
\date{%\today
}

\begin{document}

\begin{frame}{Do now}
  Answer each of the following:
  \begin{tasks}
    \task How many weeks are in a year?
    \task How many fornights are in a year?
    \task How many months are in a year?
    \task How many weeks are in a fortnight?
    \task How many weeks are in a month?
  \end{tasks}\pause
  Answers:
  \begin{tasks}(5)
    \task \pause 52
    \task \pause 26
    \task \pause 12
    \task \pause 2
    \task \pause$\approx\,$4.33
  \end{tasks}
\end{frame}

\frame{\titlepage}

\begin{frame}
  \begin{outcome}
    Success Criteria:\newline
    \textit{I can} calculate weekly, fortnightly, monthly and yearly earnings.\newline
    Activities/Tasks:
    \begin{itemize}
      \item Worksheet Ex 9B
      \item Cam Ex 2E
    \end{itemize}
  \end{outcome}
\end{frame}

\begin{frame}{Income}
  \begin{definition}[Salary]
    You are paid a set amount per year, regardless of how many hours you work.
  \end{definition}\pause
  \begin{definition}[Hourly wages]
    You are paid a certain amount per hour worked.​
  \end{definition}\pause
  \begin{definition}[Comission]
    You are paid a percentage of the total amount of sales.​
  \end{definition}
\end{frame}

\begin{frame}
  \begin{example}
    Harry works full-time and earns a salary of \$68 600 p.a. How much does he earn per:
    \begin{tasks}(3)
      \task week
      \task fornight
      \task month
    \end{tasks}
  \end{example}\pause
  \begin{solution}
    a) \[\begin{aligned}
      &\  \$68\,600\div 52 && \text{there are 52 weeks in a year}\\\pause
      =&\ \$1\,310.34 && \text{always round decimals to 2 d.p.}
    \end{aligned}\]
  \end{solution}
\end{frame}
\addtocounter{example}{-1}
\begin{frame}
  \begin{example}
    Harry works full-time and earns a salary of \$68 600 p.a. How much does he earn per:
    \begin{tasks}(3)
      \task week
      \task fornight
      \task month
    \end{tasks}
  \end{example}
  \begin{solution}
    b) \[\begin{aligned}
      &\  \$68\,600\div 26 && \text{there are 26 fortnights in a year}\\\pause
      =&\ \$2\,638.46 && \text{always round decimals to 2 d.p.}
    \end{aligned}\]
  \end{solution}
\end{frame}
\addtocounter{example}{-1}
\begin{frame}
  \begin{example}
    Harry works full-time and earns a salary of \$68 600 p.a. How much does he earn per:
    \begin{tasks}(3)
      \task week
      \task fornight
      \task month
    \end{tasks}
  \end{example}
  \begin{solution}
    c) \[\begin{aligned}
      &\  \$68\,600\div 12 && \text{there are 12 months in a year}\\\pause
      =&\ \$5\,716.67 && \text{always round decimals to 2 d.p.}
    \end{aligned}\]
  \end{solution}
\end{frame}

\begin{frame}
  \begin{example}
    Tamara earns \$2146 per week. What is her monthly salary?
  \end{example}\pause
  \vspace{-7pt}
  \begin{important}
    1 month $\neq$ 4 weeks, so we have to find Tamara's yearly salary first.
  \end{important}\pause
  \begin{solution}
    \[\begin{aligned}
      \text{Yearly Salary}=&\ 2146\times 52 && \text{there are 52 weeks in a year}\\\pause
      =&\ 111\,592\\\pause
      \text{Monthly Salary}=&\ 111\,592\div 12 && \text{there are 12 months in a year}\\\pause
      =&\ 9299.33 && \text{always round decimals to 2 d.p.}
    \end{aligned}\]
  \end{solution}
\end{frame}

\begin{frame}{Today's work}
  \begin{itemize}
    \item Worksheet - Ex 9B
    \item Cambridge Ex 2E Q3, 5, 6
  \end{itemize}
\end{frame}

\end{document}